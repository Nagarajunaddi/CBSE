\let\negmedspace\undefined
\let\negthickspace\undefined
\documentclass[journal,12pt,twocolumn]{IEEEtran}
\usepackage{gensymb}
\usepackage{amssymb}
\usepackage[cmex10]{amsmath}
\usepackage{amsthm}
\usepackage[export]{adjustbox}
\usepackage{bm}
\usepackage{longtable}
\usepackage{enumitem}
\usepackage{mathtools}
\usepackage[breaklinks=true]{hyperref}
\usepackage{listings}
\usepackage{color}                                            %%
\usepackage{array}                                            %%
\usepackage{longtable}                                        %%
\usepackage{calc}                                             %%
\usepackage{multirow}                                         %%
\usepackage{hhline}                                           %%
\usepackage{ifthen}                                           %%
\usepackage{lscape}     
\usepackage{multicol}
% \usepackage{enumerate}
\DeclareMathOperator*{\Res}{Res}
\renewcommand\thesection{\arabic{section}}
\renewcommand\thesubsection{\thesection.\arabic{subsection}}
\renewcommand\thesubsubsection{\thesubsection.\arabic{subsubsection}}
\renewcommand\thesectiondis{\arabic{section}}
\renewcommand\thesubsectiondis{\thesectiondis.\arabic{subsection}}
\renewcommand\thesubsubsectiondis{\thesubsectiondis.\arabic{subsubsection}}
\hyphenation{op-tical net-works semi-conduc-tor}
\def\inputGnumericTable{}                                 %%
\lstset{
frame=single, 
breaklines=true,
columns=fullflexible
}
\begin{document}
\newtheorem{theorem}{Theorem}[section]
\newtheorem{problem}{Problem}
\newtheorem{proposition}{Proposition}[section]
\newtheorem{lemma}{Lemma}[section]
\newtheorem{corollary}[theorem]{Corollary}
\newtheorem{example}{Example}[section]
\newtheorem{definition}[problem]{Definition}
\newcommand{\BEQA}{\begin{eqnarray}}
\newcommand{\EEQA}{\end{eqnarray}}
\newcommand{\define}{\stackrel{\triangle}{=}}
\bibliographystyle{IEEEtran}
\providecommand{\mbf}{\mathbf}
\providecommand{\pr}[1]{\ensuremath{\Pr\left(#1\right)}}
\providecommand{\qfunc}[1]{\ensuremath{Q\left(#1\right)}}
\providecommand{\sbrak}[1]{\ensuremath{{}\left[#1\right]}}
\providecommand{\lsbrak}[1]{\ensuremath{{}\left[#1\right.}}
\providecommand{\rsbrak}[1]{\ensuremath{{}\left.#1\right]}}
\providecommand{\brak}[1]{\ensuremath{\left(#1\right)}}
\providecommand{\lbrak}[1]{\ensuremath{\left(#1\right.}}
\providecommand{\rbrak}[1]{\ensuremath{\left.#1\right)}}
\providecommand{\cbrak}[1]{\ensuremath{\left\{#1\right\}}}
\providecommand{\lcbrak}[1]{\ensuremath{\left\{#1\right.}}
\providecommand{\rcbrak}[1]{\ensuremath{\left.#1\right\}}}
\theoremstyle{remark}
\newtheorem{rem}{Remark}
\newcommand{\sgn}{\mathop{\mathrm{sgn}}}
\providecommand{\abs}[1]{\left\vert#1\right\vert}
\providecommand{\res}[1]{\Res\displaylimits_{#1}} 
\providecommand{\norm}[1]{\left\lVert#1\right\rVert}
%\providecommand{\norm}[1]{\lVert#1\rVert}
\providecommand{\mtx}[1]{\mathbf{#1}}
\providecommand{\mean}[1]{E\left[ #1 \right]}
\providecommand{\fourier}{\overset{\mathcal{F}}{ \rightleftharpoons}}
%\providecommand{\hilbert}{\overset{\mathcal{H}}{ \rightleftharpoons}}
\providecommand{\system}{\overset{\mathcal{H}}{ \longleftrightarrow}}
	%\newcommand{\solution}[2]{\textbf{Solution:}{#1}}
\newcommand{\solution}{\noindent \textbf{Solution: }}
\newcommand{\cosec}{\,\text{cosec}\,}
\providecommand{\dec}[2]{\ensuremath{\overset{#1}{\underset{#2}{\gtrless}}}}
\newcommand{\myvec}[1]{\ensuremath{\begin{pmatrix}#1\end{pmatrix}}}
\newcommand{\mydet}[1]{\ensuremath{\begin{vmatrix}#1\end{vmatrix}}}
\numberwithin{equation}{subsection}
\makeatletter
\@addtoreset{figure}{problem}
\makeatother
\let\StandardTheFigure\thefigure
\let\vec\mathbf
\renewcommand{\thefigure}{\theproblem}
\def\putbox#1#2#3{\makebox[0in][l]{\makebox[#1][l]{}\raisebox{\baselineskip}[0in][0in]{\raisebox{#2}[0in][0in]{#3}}}}
     \def\rightbox#1{\makebox[0in][r]{#1}}
     \def\centbox#1{\makebox[0in]{#1}}
     \def\topbox#1{\raisebox{-\baselineskip}[0in][0in]{#1}}
     \def\midbox#1{\raisebox{-0.5\baselineskip}[0in][0in]{#1}}
\vspace{3cm}
\title{CBSE MATHEMATICS 2020}
\author{G V V Sharma$^{*}$
	\thanks{}
}
\maketitle
\newpage
\bigskip
\renewcommand{\thefigure}{\theenumi}
\renewcommand{\thetable}{\theenumi}
\section{Section-A}
\textbf{Question numbers 1 to 20 carry 1 mark each.} \\
\textbf{Question numbers 1 to 10 are multiple choice type questions. Select the correct option.} \\
\renewcommand{\theequation}{\theenumi}
\begin{enumerate}[label=\thesection.\arabic*.,ref=\thesection.\theenumi]
\numberwithin{equation}{enumi}
\item The area of a triangle formed by vertices O, A and B, where $\overrightarrow{OA}=\hat{i} + 2\hat{j} + 3 \hat{k} $ and $\overrightarrow{OB}=-3\hat{i} - 2\hat{j} +  \hat{k} $ is \\
 \begin{enumerate}
     \item $ 3\sqrt{5} $ sq.units\\
    \item$ 5\sqrt{5} $ sq.units\\
    \item$ 6\sqrt{5} $ sq.units\\
    \item$ 4 $ sq.units
\end{enumerate}

\item If $\cos (\sin^{-1} \frac{2}{\sqrt 5} + \cos^{-1} x)$ 

\begin{enumerate}
   \item $\frac{1}{\sqrt5}$\\
   \item $ \frac{-2}{\sqrt5}$\\
    \item $ \frac{2}{\sqrt5}$\\
    \item $1$
\end{enumerate}

\item The interval in which the function f given by $ f({x}) = x^2e^{-x} $ is strictly increasing,  is

\begin{enumerate}
    \item $\left(\infty , -\infty \right)$
    \item $\left(\infty , 0 \right)$
    \item $\left(2 , \infty \right)$
    \item $\left(0 , 2 \right)$
\end{enumerate}
\item The function $ f\left({x} \right) = \frac{x-1}{x\left(x^2 -1 \right)} $ is discontinuous at

\begin{enumerate}
    \item Exactly one point
    \item Exactly two points
    \item Exactly three points
    \item No point
\end{enumerate}

\item The function  $ f : R \rightarrow \left[-1,1 \right] $ defined by $ f\left(x \right) = cosx $ is

\begin{enumerate}
    \item Both one-one and onto
    \item Not one-one, but onto
    \item one-one, but Not onto
    \item Neither one-one, nor onto
\end{enumerate}

\item  The coordinates of the foot of the perpendicular drawn from the point $ \left(2,-3,4 \right) $ on the y-axis is 

\begin{enumerate}
    \item $\left(2,3,4\right)$
    \item $\left(-2,-3,-4\right)$
    \item $\left(0,-3,0\right)$
    \item $\left(2,0,4\right)$
\end{enumerate}

\item  The relation R in the set $ \{1,2,3\}$  given by $R=\{(1,2)(2,1)(1,1)\}$ is

\begin{enumerate}
    \item Symmetric and transitive, but not reflexive  \\
    \item reflexive and symmetric, but not transitive  \\
    \item Symmetric, but neither reflexive nor transitive \\
    \item An equivalence relation 
\end{enumerate}
\item  The angle between the vectors $ \hat{i} - \hat{j} $ and $ \hat{j} - \hat{k} $ is

\begin{enumerate}
    \item $\frac{-\pi}{3}$
    \item 0
    \item $\frac{\pi}{3}$
    \item $\frac{2\pi}{3}$
\end{enumerate}

\item  If A is a non-singular square matrix of order 3 such that $ A^2 =3A $, then value of  $\begin{vmatrix}A \end{vmatrix}$ is

\begin{enumerate}
    \item -3
     \item 3
     \item 9
     \item 27
\end{enumerate}
\item  If $\begin{vmatrix}\overrightarrow{a} \end{vmatrix} = 4 $ and  $ -3\leq \lambda \leq 2 $ then $\begin{vmatrix}\lambda \overrightarrow{a} \end{vmatrix} $ lies in

\begin{enumerate}
    \item $\left[0,12\right]$
    \item $\left[2,3\right]$
    \item $\left[8,12\right]$
    \item $\left[-12,8\right]$
\end{enumerate}
\pagebreak
\textbf{Fill in the blanks in question number 11 to 15}\\

\item  If the radius of the circle is increasing at the rate of 0.5cm/s, then the rate of increase of its circumference is ---------------\\
\item  If $\begin{vmatrix} 2x & -9 \\ -2 & x \end{vmatrix}$ = $\begin{vmatrix} -4 & 8 \\ 1 & -2 \end{vmatrix}$ , then value of x is -------------\\
\item  The corner points of the feasible region of an LPP are (0,0),(0,8),(2,7),(5,4) and (6,0). The maximum profit P=3x+2y occurs at the point .................\\
\item  The range of the principle value branch of the function $ y= \sec^{-1}x $ is ----------------
    \begin{center}
        OR
    \end{center}
The principal value of $\cos^{-1} \left(\frac{-1}{2}\right)$ is --------------\\
\item  The distance between parallel planes 2x+y-2z-6=0 and 4x+2y-4z=0 is ------------------ units.
\begin{center}
        OR
    \end{center}
    If P(1,0,-3) is the foot of the perpendicular from the origin to the plane, then the Cartesian equation of the plane is .................... \\
    
\textbf{Question numbers 16 to 20 are very short answer type questions}\\  
\item  Evaluate :
        \begin{center} 
        $\int_{-\frac{\pi}{2}}^{\frac{\pi}{2}} x \cos^2 x dx$ 
        \end{center}  

\item  Find the coordinates of the point where the line 
        \begin{center}
        $\frac{x-1}{3}$ = $\frac{y+4}{7}$ = $\frac{z+4}{2}$
        \end{center} 
        cuts the xy-plane. \\

\item  Find the value of k, so that the function \begin{equation*}  f(x)  = \begin{cases}
                $ k$ x^2 $+ 5 $ ,  &\text{ if  x $\leq$ 1 } \\
        2 , &\text{if x $ > $ 1}
\end{cases} \end{equation*}  is continuous at x=1. \\
\item  Find the integrating factor of the differential equation \begin{center} $ x\frac{dy}{dx} = 2x^2 +y $ \end{center} .
\item  Differentiate $ \sec^2(x^2) $ with respect to $ x^2 $.
    \begin{center}
        OR
    \end{center}
 If $y=f(x^2)$  and  f'(x)= $ e^{\left(\sqrt{x}\right)}$  , then find $\frac{dy}{dx}$.\\

\end{enumerate}
\section{Section-B}

\textbf{Question numbers 21 to 26 carry 2 marks each.}\\

\renewcommand{\theequation}{\theenumi}
\begin{enumerate}[label=\thesection.\arabic*.,ref=\thesection.\theenumi]
\numberwithin{equation}{enumi}
\item  Find a vector $\overrightarrow{r}$ equally inclined to the three axes and whose magnitude is $3\sqrt{3}$ units.
 
    \begin{center}
        OR
    \end{center}
    
    Find the angle between unit vectors $\overrightarrow{a}$ and $\overrightarrow{b}$ so that $\sqrt{3}$ $\overrightarrow{a}$ - $\overrightarrow{b}$ is also a unit vector.\\
    
\item If $A=\begin{bmatrix}-3 & 2 \\ 1 & -1 \end{bmatrix} $ and $ I=\begin{bmatrix}1 & 0 \\ 0 & 1 \end{bmatrix}$, Find scalar k so that $A^2 + I = kA$.\\

        \item If $f\left(x\right)$= $\sqrt{\frac{sec x-1}{sec x+1}}$, Find $ f'\left(\frac{\pi}{3}\right).$
\begin{center}
        OR
    \end{center}
Find $f'$ $\left(x\right)$ if $f\left(x\right)=\left(\tan x\right)^{\left(\tan x\right)}$.\\

\item Find : \begin{center} $\int \frac{\tan^3x}{\cos^3x} dx $ \end{center} 

\item Show that the plane $x-5y-2z=1$ contains the line $\frac{x-5}{3}=y=2-z$.\\

\item A fair dice is thrown two times. Find the probability distribution of the number of sixes. Also determine the mean of the number of sixes.\\
\end{enumerate}
\section{Section-C}

\textbf{Question numbers 27 to 32 carry 4 marks each.}\\

\renewcommand{\theequation}{\theenumi}
\begin{enumerate}[label=\thesection.\arabic*.,ref=\thesection.\theenumi]
\numberwithin{equation}{enumi}
\item Solve the following differential equation: $(1+e^{\frac{y}{x}}) dy+e^{\frac{y}{x}(1-\frac{y}{x})} dx = 0 $  $(x\not= 0)$

\item A cottage industry manufactures pedestal lamps and wooden shades. Both the products require machine time as well as craftsman time in the making. The number of hours required for producing 1 unit of each and the corresponding profit is given in the following table :\\

%\begin{center}
\resizebox{\columnwidth}{!}{
\begin{tabular}{|c|c|c|c|}
\hline
% \begin{tabularx}{\linewidth} {lX}
 Item & Machine Time & Craftsman Time & Profit(in INR) \\ 
 \hline
 Pedestal Lamp & 1.5 hours & 3 hours & 30\\  
 \hline
 Wooden shades & 3 hours & 1 hour & 20 \\
 \hline
%  \end{tabularx}
\end{tabular}
}
%\end{center}
\\
In a day, the factory has availability of not more than 42 hours of machine time and 24 hours of craftsman time.\\
Assuming that all items manufactured are sold, how should the manufacturer schedule his daily production in order to maximise the profit? Formulate it as an LPP and solve it graphically.\\

 \item Evaluate : $\int_{0}^{\frac{\pi}{2}}$ $\sin2x \tan^{-1}$  $\left(\sin x\right)dx$\\

\item Check whether the relation R in the set N set of natural numbers given by R= \{(a,b):a is divisor of b\} is reflexive, symmetric or transitive. Also determine whether R is an equivalence relation. 

\begin{center}
        OR
    \end{center}
    
    Prove that $\tan^{-1}$ $\frac{1}{4}$ + $\tan^{-1}$ $\frac{2}{9}$ = $\frac{1}{2}$ $\sin^{-1}$ $(\frac{4}{5})$.\\

\item Find the equation of the plane passing through the points $(1,0,-2)$ , $(3,-1,0)$ and perpendicular to the plane $ 2x-y+z=8 $. Also find the distance of the plane thus obtained from the origin.\\

\item If $\tan^{-1}(\frac{y}{x})=\log \sqrt{x^2+y^2}$ , prove that $\frac{dy}{dx}=\frac{x+y}{x-y}$.  

\begin{center}
        OR
    \end{center}
    
   If  $ y=e^{(acos^{-1}x)} $ , $-1 < x < 1$ , then show that  $(1-x^2) \frac{d^2y}{dx^2} - x \frac{dy}{dx} - a^2y =0 $\\

 \section{Section-D}
 
 \textbf{Question numbers 33 to 36 carry 6 marks each.}\\
 
 \item Amongst all open (from the top) right circular cylindrical boxes of volume $125 \pi $ $cm^3$, find the dimensions of the box which has the least surface area.\\  
 
\item Using integration, Find the area lying above x-axis and included between the circle $ x^2 + y^2 =8x $ and inside the parabola $ y^2 =4x $.   

\begin{center}
        OR
    \end{center}
    
  Using the method of integration, find the area of the triangle ABC, coordinates of whose vertices are A(2,0), B(4,5) and C(6,3).\\

\item If $A=\begin{bmatrix}5 & -1 & 4 \\ 2 & 3 & 5 \\ 5 & -2 & 6 \end{bmatrix} $, Find $A^{-1}$ and use it to solve the following system of the equations: \\
\begin{center}
        $5x-y+4z=5 $\\
        $2x+3y+5z=2 $\\
        $5x-2y+6z=-1 $
    \end{center}
    
    \begin{center}
        OR
    \end{center}
    \vspace{0.3in}
If x,y,z are different and $\begin{vmatrix}x & x^2 & 1+x^3 \\ y & y^2 & 1+y^3 \\ z & z^2 & 1+z^3 \end{vmatrix} =0 $, then using properties of determinants show that $1+xyz=0.$\\

\item A card from a pack of 52 cards is lost. From the remaining cards of the pack, two cards are drawn randomly one-by-one without replacement and are found to to be both kings. Find the probability of the lost card being a king.\\
\end{enumerate}
\end{document}
